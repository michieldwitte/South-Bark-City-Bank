\documentclass[10pt,a4paper]{report}
\usepackage[latin1]{inputenc}
\usepackage{amsmath}
\usepackage{amsfonts}
\usepackage{amssymb}
\author{Sander Demeester}
\title{Password storage crypto-design.}
\begin{document}
\begin{subsection}{Client-side}
$U_1$ = Username \\
$P_1$ = Wachtwoord \\
We maken gebruik van PBKDF2 om op de client-side het wachtwoord te hashen.
$S_{1} = PBKDF2((P_1 || U_1), H(P_1 || U_1),4096)$.
Daarna versturen we het koppel $(U_1,S_1)$ naar de server.
\end{subsection}
\begin{subsection}{Server-side}
Op de server ontvangen we $(U_1,S_1)$.
Het eerste wat we doen is een random salt $S_2$ maken. Deze salt zullen we samen met $S_1$ door een scrypt functie laten lopen om een AES-key  te maken.\\ 
$K_1 = scrypt(S_1,S_2)$. \\
Deze key is uniek per user en dient enkel om specefieke informatie dat is geassocieerd met de user zijn acccount te beveiligen.
. 
Als dit de eerste keer is dat een account word gemaakt zal hier nu ook een seqentie key worden aangemaakt die zal worden gebruikt als input voor onze OTP. Er zal een 64bit counter zijn die hier ook zal worden opgeslagen onder invloed van het AES-cijfer.
Daarnaast maken we nu nog een salt $S_3$ die samen met $S_1$ zal worden gehased en zal worden opgeslagen geencrypteerd in de databank. Het is deze hash die zal worden gebruikt om te kijken of een user het recht heeft om in te loggen.
Volgende tabel maakt dit allemaal wat duidelijker.
\begin{tabular}{|c|c|}
\hline 
Encrypteerd & Plain-text \\ 
\hline 
Hash($S_1$,$S_3$) & $U_1$ \\ 
\hline 
Seqentie string & $S_2$ \\ 
\hline 
Counter &  \\ 
\hline 
Andere bank info &  \\ 
\hline 
\end{tabular} 
De sequentie key en de counter zullen moeten worden gecommuniceerd met de client-side (android app). 
\end{subsection}
\begin{subsection}{Het verantwoorden van mijn keuze}
Het is belangrijk op te merken dat de data die we op de server bijhouden enkel kan worden gedecrypteerd als de user is ingelogd. We hebben namelijk $S_1$ nodig om de AES key terug te kunnen maken en de user zijn account informatie te decrypteren, de eenige manier dat wij $S_1$ kunnen krijgen is via de user. De reden dat we salt $S_2$ nodig hebben is zodat als de user controle verliest over $S_1$ (dit kan enkel als een virus in de user zijn browser zit, of de SSL verbinding word gesnift) dan kan de aanvaller nog altijd niet aan de user zijn persoonlijke info want hij heeft geen bezit van $S_2$ die uniek is per user. Het senario hier zou zijn dat een aanvaller de user zijn info nodig heeft EN onze databank moet gehackt zijn. De volgende tabel maakt dit duidelijk:
\begin{tabular}{|c|c|}
\hline 
Compromised & Gevolg \\ 
\hline 
User zijn gebruikers naam en wachtwoord & De aanvaller kan nog niet inloggen omdat hij nog een OTP nodig heeft \\ 
\hline 
Database gehackt & De aanvaller kent $S_1$ niet dus kan de AES-key niet vinden. Dus kan niet aan user info \\ 
\hline 
OTP android app gehackt & De aanvaller kent de user zijn password en username niet \\ 
\hline 
Rainbow tables & Er word een salt gebruikt om de client-side en op de serer side met KDF functies die het computationeel bijna onmogelijk maakt rainbow tables te maken \\ 
\hline 
\end{tabular} 
De reden dat ik heb gekozen voor scrypt is omdat het onmogelijk is dit te paraliseren. Het maakt gebruik van een memory-hard algoritme. Het gebruik hier van GPU of paralelle algoritmes heeft hier geen zin. De salt die hier word gebruikt is uniek per user. 
Om de user zijn hash te controlleren moeten we met de AES-key zijn personal blob decrypteren en $S_3$ gebruiken om een hash te maken van $S_1$ en $S_3$. Nu hebben we ook toegang tot alle info om de user zijn OTP te controlleren.
Info:www.tarsnap.com/scrypt/scrypt.pdf
\end{subsection}
\end{document}