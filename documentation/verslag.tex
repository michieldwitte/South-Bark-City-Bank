\documentclass[11pt]{article}
\usepackage{hyperref}
\setlength{\parindent}{0cm}
%Gummi|061|=)
\title{\textbf{Beveiliging voor online banking}}
\author{Sander Demeester\\
		Michiel De Witte\\
		Stef Trenson}
\date{}
\begin{document}

\maketitle

\section{Inleiding}
Online bankieren en bekijken van Financi\"ele informatie op een computer is onmisbaar in de digitale wereld van vandaag. Goeie beveiligingsmechanisme zijn dus van een cruciaal belang voor deze system. 

\section{Beveiligingseisen}
\begin{itemize}
\item Privacy (tussen de gebruikers, de bank zelf moet kunnen verificeren dat het echt wel de juiste gebruiker is). De gebruiker moet kunnen vertrouwen dat de bank zijn/haar informatie geheim houd. Het limeteren van wie in de bank aan welke informatie kan is hier van groot belang.
\item Vertrouwelijkheid (encryptie van informatie)
\item Verificatie (is de informatie wel correct?)
\item Authenticatie (praat je wel met de juiste server, praat de server wel met wie hij denkt te praten?)
\item Worden betalingen en transacties correct uitgevoerd. Is er niet repudiation?
\item Systeem vertrouwen/persoonvertrouwen (vertrouw je iedereen die werkt in uw bank met uw gegevens).
\end{itemize}
\subsection{Beveiliging aanvalsvectoren}
%TODO: Hier bekijken we het threat model. Waar beschermen we tegen en vooral waar beschermen we niet tegen
We analyseren het
\subsection{Preventie beveiliging aanvalsvectoren}
\section{Implementatie}
\section{Conclusie}
If you are wondering where your old default text is; it has been stored as a template. The template menu can be used to access and restore it. 

\end{document}
