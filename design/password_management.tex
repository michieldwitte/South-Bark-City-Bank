\documentclass[10pt,a4paper]{report}
\usepackage[latin1]{inputenc}
\usepackage{amsmath}
\usepackage{amsfonts}
\usepackage{amssymb}
\author{Sander Demeester}
\title{Password storage crypto-design.}
\begin{document}
\begin{subsection}{Client-side}
Om onze wachtwoorden veilig op te slaan maken we gebruik van het volgende algoritme.
\begin{subsection}{Client-side}
Langs de client-side hebben we de username $U$ en het wachtwoord $P$. We concateneren dit tot $S = U||P$, dit is de client-side salt. Het wachtwoord gaat samen met de salt $S$ en scrypt iteratief beveiligen. Scrypt is het equiv van bcrypt maar is een memory hard algoritme. De output daarvan noemen we $Sc$ word samen met de username doorgestuurd naar de server.
\end{subsection}
\begin{subsection}{Server-side}
De server ontvangt het paar $(Sc,U)$. De server maakt een random salt aan die, samen met $Sc$ wordt gebruikt om een encryptie key te maken voor de user zijn sequence key en zijn persoonlijke informatie op te slaan. We noemen deze salt $Salt_{1}$. 
$DK = PBKDF2(PRF, Sc,Salt_{1},c,256)$. Nu maken we een random sequence key aan te zal worden gebruikt om onze OTP mee aan te sturen, we noemen dit de OTP-sessie key. We slaan $Sc$, OTP-sessie key en een couter op in de databank geencrypteerd met het AES-cijfer met als input key $DK$. 
\end{subsection}
\end{document}