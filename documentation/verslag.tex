\documentclass[11pt]{article}
\usepackage{fullpage}

\usepackage{hyperref}
\setlength{\parindent}{0cm}
%Gummi|061|=)
\title{\textbf{Beveiliging voor online banking}}
\author{Sander Demeester\\
		Michiel De Witte\\
		Stef Trenson}
\date{}
\begin{document}

\maketitle

\section{Inleiding}
Online bankieren en bekijken van Financi\"ele informatie op een computer is onmisbaar in de digitale wereld van vandaag. Goeie beveiligingsmechanisme zijn dus van een cruciaal belang voor deze system. 

\section{Beveiligingseisen}
\begin{itemize}
\item Privacy (tussen de gebruikers, de bank zelf moet kunnen verificeren dat het echt wel de juiste gebruiker is). De gebruiker moet kunnen vertrouwen dat de bank zijn/haar informatie geheim houd. Het limeteren van wie in de bank aan welke informatie kan is hier van groot belang.
\item Vertrouwelijkheid (encryptie van informatie)
\item Verificatie (is de informatie wel correct?)
\item Authenticatie (praat je wel met de juiste server, praat de server wel met wie hij denkt te praten?)
\item Worden betalingen en transacties correct uitgevoerd. Is er niet repudiation?
\item Systeem vertrouwen/persoonvertrouwen (vertrouw je iedereen die werkt in uw bank met uw gegevens).
\end{itemize}
We maken het onderscheid tussen 2 belangrijke zaken. nl: 
\begin{itemize}
\item vertrouwelijkheid van gegevens
\item Authenticatie van entiteiten met een zoon hoog mogelijke zekerheid van identificatie van beide actoren (server,client).
\end{itemize}
\subsection{Beveiliging aanvalsvectoren}
%TODO: Hier bekijken we het threat model. Waar beschermen we tegen en vooral waar beschermen we niet tegen
We analyseren het \" threat model \" van ons systeem. Aan de hand van onze applicatie-eisen zullen we een threat model opstellen en proberen mogelijke aanvalsvectoren te identificeren.
\subsubsection{Authenticatie van entititen op applicatie niveau}
%TODO: bespreken wachtwoorden en andere mechanismes voor authenticatie en verificatie
%Opnemen van multifactor authenticatie en hun nut (something you are, something you know, something you have).
Waarom wachtwoorden? Wat bewijzen wachtwoorden? Waarom zijn wachtwoorden een zwake vorm van authenticatie? Het belangt van OTP en het belang van een niet statische 2de factor.

\subsubsection{Denial of Service}
%TODO: bespreken van DoS aanvallen op server

\subsubsection{Vertrouwelijkheid op applicatie niveau}
%TODO: bespreken van encryptie in de databank. 
Bespreken van bcrypt, scrypt. Memory hard problems. Waarom encrypteren op de client en op de server kant (omdat de bank nooit zelf zou beschikken over het originele wachtwoord dat de gebruiker elder zou hergebruiken).

\subsubsection{Vertrouwelijkheid op transport niveau}
%TODO: bespreken van TLS/SSL en het probleem met MITM en CBC BEAST-attack op ssl3.0/tls1.0 (opgelost in TLS1.1 maar geen goeie adoptie (gebruik van RC4 is hier een oplossing)
% bespreken van compressie probleem bij TLS (Compression Ratio Info-leak Made Easy attack)
TLS/SSL is de manier om vandaag vertrouwelijkheid te realiseren op transport/netwerk laag. We kunnen 2 groepen van problemen onderscheiden.

\begin{itemize}
\item Menselijke fouten
\item Implementatie/algoritmische fouten
\end{itemize}

Menselijke fouten is het negeren van browser waarschuwingen en phishing emails niet te negeren.
Implementatie/algoritmische fouten zijn fouten die geintroduceerd zijn door het beveiligingssysteem zelf of door slechte configuratie van het systeem. Zoals het gebruik van md5 voor digital signatures bij ssl/tls (md5 second-preimage resistance is gebroken).
Daarnaast is er een probleem bij het gebruikt van CBC-encryption mode bij ssl3.0/tls1.0. CBC mode in ssl2.0/tls1.0 maakt gebruik van chained initialization vectors die het mogelijk maken een MITM aanval uit te voeren door een blockwise chosen-boundary aaval uit te voeren. Het resultaat is dat HTTP headers leesbaar worden en een deel vertrouwelijkheid verloren gaat. \\

Daarnaast is het mogelijk voor een hacker om session-hijacking te doen op een ssl/tls verbinding als de verbinding compressie gebruikt. Bij deze aanval gaat niet enkel de vertrouwelijkheid verloren maar ook de authenticiteit van beide eindpunten is verloren.

\subsection{Mechanismes om te voldoen aan Beveiligingseisen}
op welke manieren probeert ons syteem te voldoen aan de beveiligingseisen die worden gesteld.

\subsubsection{Multi-factor authentication}
Dit is de notie van \"something you have and something you know \". 
Het traditioneel gebruik van wachtwoorden is niet voldoende. Wachtwoorden worden hergebruikt en zijn een single point of failure voor een authenticatie systeem. We moeten dus gebruik maken van een extra factor waar beide entititen (zowel de bank als de klant) weet van moeten hebben. Het is belangrijk dat deze factor onafhankelijk is van het systeem en niet berust op de gebruiker om iets te moeten onthouden (zoals een wachtwoord). Deze werkwijze neemt meestal de vorm aan van een hardware token.
\section{Architectuur en Implementatie}
\subsection{Architectuur}
\subsection{Implementatie}
\section{Conclusie}
If you are wondering where your old default text is; it has been stored as a template. The template menu can be used to access and restore it. 

\bibliography{information_security_bibtex}{}
\bibliographystyle{plain}
\end{document}
