\documentclass[10pt,a4paper]{report}
\usepackage[latin1]{inputenc}
\usepackage{amsmath}
\usepackage{amsfonts}
\usepackage{hyperref}
\usepackage{amssymb}
\author{Sander Demeester}
\title{OTP-crypto design}
\begin{document}
\begin{section}*{HMAC-based one-time password algoritme}
\url{http://tools.ietf.org/html/rfc4226}
\href{http://tools.ietf.org/html/rfc4226}{HOTP: An HMAC-Based One-Time Password Algorithm}\\
\url{http://tools.ietf.org/html/rfc6238}
\href{http://tools.ietf.org/html/rfc6238}{TOTP: Time-Based One-Time Password Algorithm}
\section*{HOTP}
HMAC based one-time password algoritme voor 2-factor authenticatie.
\subsection*{Definitie}
\begin{itemize}
\item $K$ is een \emph{secret key}.
\item $C$ is een \emph{counter}.
\item HMAC($K$,$C$) is een HMAC functie, in rfc4226 staat vermeld dat hier SHA1 moet worden gebruikt. 
\item Selecteer 4 bytes van de resulterende HMAC. We noemen deze functie $T()$. De 4-bytes moeten altijd op de zelde manier worden gekozen.
\item HOTP($K$,$C$) = $T($HMAC($K$,$C$)) \& 0x7FFFFFFF. De reden dat we hier een AND-mask toepassen is om de MSB weg te werken enzo meer compatibel te zijn tussen verschillende processen.
\end{itemize}
\subsection*{Voorwaarden}
Zoals vermeld moet er worden gebruikt gemaakt van een \emph{secret key}. Deze key moet gekend zijn door de client als door de server, alsook de counter. De counter moet nooit worden gecommuniceerd met de server. De \emph{secert key} wel.\\

Eventueel kunnen we hier TOTP van maken door unix-time te gebruiken.
\end{section}
\end{document}